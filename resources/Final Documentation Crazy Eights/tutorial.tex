
\section{\LaTeX ``Tutorial''}

OMIT this section in production, but here is some sample \LaTeX~to munch on.

Use the table environment to include tables.
Be sure to keep the label and caption below the tabular!

Setting up tables can be annoying. You can also set it up in excel and then
copy it into \textsf{https://www.tablesgenerator.com} to generate the latex. 



\begin{table}[h!]
    \centering
    \begin{tabular}{|c|c|c|c|c|}
    \hline
      Risk   & Impact & Priority  & Overall Rating  & Mitigation \\
         &  &  &  & \\\hline
         &  &  &  & \\\hline
         &  &  &  & \\\hline
    \end{tabular}
    \caption{Risks to the software.}
    \label{tab:risks}
\end{table}

Table~\ref{tab:risks} overviews the risk relevant to our project.


Use the figure environment to include images such as \textsf{.png} in your latex document. \\

\begin{figure}[!bh]
\centering
\includegraphics[width=0.33\textwidth]{figures/sb.png} \\
\caption{Class Diagram for Sponge Bob's Best Christmas Ever.}
\label{fig:class}
\end{figure}

Figure~\ref{fig:class} shows the high-level class diagram.


\begin{figure}[!bht]
\begin{lstlisting}[language=Java]
public static void main(String [] args)
{
  System.out.println("Hello World!");
}
\end{lstlisting}
\caption{Hello Example.}
\label{fig:hello}
\end{figure}

Figure~\ref{fig:hello} shows the code used to print hello.
\textbf{Note that you writeup should include code sparingly.
The details of the code are not its focus.}

