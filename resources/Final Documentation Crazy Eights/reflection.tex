\section{Reflections on the Project}

We learned a lot of lessons over the course of this project. One of the main things we learned was how to work and communicate effectively as a team. Our team did not have any specific leader, we did not need one because we each trusted each other to take on an equal amount of work and get things done when they needed to be done. We were well organized, and while we worked both separately and synchronously, it was never an issue as we communicated frequently. We did not have any notable disagreements that hindered progress.

The main strength of our group's execution of the project is that we each took on a specific software development role when building the website. Will worked on the database end, Jack worked on the server end, and Victoria worked on the front end. This allowed each of us to become an expert in our own topic, and assist other members if they needed to do something in a different area than their specialty. It also gave us a good way to divide up work between the group. 

The main barriers were our limited time and experience. Three two week sprints did not give us enough time, especially since we had to learn Javascript, React, and other topics relating to building and hosting a website. While three sprints may have been enough if we focused solely on the project, each of us had demanding classes and other priorities outside of class, forcing us to either sacrifice other classes or our completion of the project. While our website was perfectly functional in the end, it lacked the polish it could have had if we were able to work on it for a longer amount of time. If we had the knowledge we did today about the difficulties of the process, we could have done a few things differently. We could have prepared more for the project by learning the necessary topics beforehand, and spread work for each sprint throughout the week.

We learned a lot about working with the Agile process of software development. The sprints were a good layout for the project, but could have used either more time or a larger quantity of sprints. The Kanban board was very useful for keeping track of tasks and assigning them to each person. The user stories were difficult for the team to understand and work with, as we did not know how to format them as tasks to complete. Overall, the project was very informative, but a bit too much for the limited amount of time we had.



\subsection{AI Impact}
Each team member should write a paragraph or two on their experience using LLMs/AIs. If
you didn't make use of such tools, then write why you made that choice.

\subsubsection{Will}
My use of AI was mainly facilitated around being able to quickly learn new syntax given I was writing in a new language. Having never written in javascript, html or css before, I knew what I needed to do but the syntax was completely foreign to me which made it very hard to start. At the beginning I was able to use it to learn the syntax for classes, functions and other generic concepts that are consistent across programming languages. By doing this I was able to very quickly learn the syntax of all of the concepts that I would need for the project. Additionally, it was very helpful for transitioning my existing code from SQL to NOSQL when we made the switch to firebase. The AI was able to generate a basic structure for interacting with the database that I was able to modify as needed for each individual database function.

\subsubsection{Jack}
I used AI for a few things. Firstly, I used AI to learn React, JavaScript, HTML/CSS and SocketIO quickly. My experience with AI is that it gives incorrect or inefficient code, but it is great at giving the user a basic understanding of a topic. For example, when I wanted to learn JavaScript, I would ask how do var, const and let work. This would accelerate my learning, but it didn't give me much code. I also asked AI how to change the users page through code. That led me to finding the navigate function. The other way that I used AI was to debug my code. Sometimes, I would have an error, but I couldn't find it. So, I would ask AI to tell me where the bug is. This was inconsistent, but it saved me some time when it worked. 

\subsubsection{Victoria}

My use of AI was mainly to accelerate my learning of React, JavaScript, Jest, and CSS's syntax. Due to the time constraints of the project and my unfamiliarity with full stack development, I had to take shortcuts to learn how to complete my user stories on time. 

I used AI to give me a very quick and rough idea of what a certain section of code would look like, such as how a button executing a function would be implemented, what a class in CSS looked like to create a box in a certain area of the screen, how to display an image or video, how to write a test in Jest for a component in React, etc. This allowed me to comprehend the syntax much faster than checking documentation or tutorials and learning by trial and error. Once I had seen an example of how a section of code would look like, I was able to understand it, build on it, and use the topics in other sections of the project without the assistance of AI. 

I also used AI in the case of writing tests, as writing hundreds of tests by hand would have been far too time consuming. However, I spent a lot of time debugging the AI's attempts at writing tests, so it may have taken equal or less time to write the tests myself.